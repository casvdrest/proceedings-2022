\newcommand{\LambdaszType}{
\begin{figure}

\lefttext{Syntax of Types and Effects}
\begin{align*}
A, B &::= \basety \mid \carrowty{A}{C} && \text{Value Types} \\
C, D, E &::= \cty{A}{\epsilon} && \text{Computation Types} \\
\epsilon &::= \mteff \mid \conseff{A}{\epsilon} && \text{Effects}
\end{align*}

\lefttext{Typing Rules}

\[
\typing{x : A \in \Gamma}
       {\judgev{\Gamma}{x}{A}}
       {Var}
\qquad
\typing{\judgec{\extenv{\Gamma}{x}{A}}{M}{C}}
       {\judgev{\Gamma}{\abs{x}{M}}{\carrowty{A}{C}}}
       {Abs}
\qquad
\typing{\judgev{\Gamma}{V}{A}}
       {\judgee{\Gamma}{\ret{V}}{A}{\epsilon}}
       {Return}
\]

\[
\typing{\judgev{\Gamma}{V}{\carrowty{A}{C}} \quad
        \judgev{\Gamma}{W}{A}}
       {\judgec{\Gamma}{\app{V}{W}}{C}}
       {App}
\qquad
\typing{\judgee{\Gamma}{M}{A}{\epsilon} \quad
        \judgee{\extenv{\Gamma}{x}{A}}{N}{B}{\epsilon}}
       {\judgee{\Gamma}{\letin{x}{M}{N}}{B}{\epsilon}}
       {Let}
\]

\[
\typing{\judgee{\extenv{\Gamma}{k}{\earrowty{A}{B}{\epsilon}}}
               {M}{B}{\epsilon}}
       {\judgee{\Gamma}{\shiftz{k}{M}}{A}{(\conseff{B}{\epsilon})}}
       {Shift0}
\qquad
\typing{\begin{trgather}
        \judgee{\Gamma}{M}{A}{(\conseff{B}{\epsilon})} \\
        \judgee{\extenv{\Gamma}{x}{A}}{N}{B}{\epsilon}
        \end{trgather}}
       {\judgee{\Gamma}{\dollarret{M}{x}{N}}{B}{\epsilon}}
       {Dollar}
\]

\caption{Type System of \lambdasz}
\label{fig:lambdasztype}
\end{figure}
}

\newcommand{\LambdahType}{
\begin{figure}

\lefttext{Syntax of Types and Effects}
\begin{align*}
A, B &::= \basety \mid \carrowty{A}{C} && \text{Value Types} \\
C &::= \cty{A}{\epsilon} && \text{Computation Types} \\
\epsilon &::= \pureeff \mid \effp{A}{B} && \text{Effects} \\
\end{align*}

\lefttext{Typing Rules}

\[
\typing{\judgev{\Gamma}{V}{A}}
       {\judgee{\Gamma}{\doop{V}}{B}{\effp{A}{B}}}
       {Do}
\qquad
\typing{\begin{trgather}
        \judgee{\Gamma}{M}{A}{\effp{\Apr}{\Bpr}} \\
        \judgec{\extenv{\Gamma}{x}{A}}{M_r}{C} \\
        \judgec{\extenvtwo{\Gamma}{x}{\Apr}{k}{\carrowty{\Bpr}{C}}}
               {M_h}{C}
        \end{trgather}}
       {\judgec{\Gamma}{\handle{M}{x}{M_r}{x}{k}{M_h}}{C}}
       {Handle}
\]

\caption{Type System of \lambdah
(rules for $\lambda$-terms are the same as Figure~\ref{fig:lambdasztype})}
\label{fig:lambdahtype}
\end{figure}
}

\newcommand{\TyEff}{
\begin{figure}

\begin{align*}
A, B &::= \basety \mid \carrowty{A}{C} && \text{Value Types} \\
C, D &::= \cty{A}{C} && \text{Computation Types} \\
\epsilon &::= \pureeff \mid \effc{A}{B}{C} && \text{Effects}
\end{align*}

\caption{Syntax of \lambdah Effects and Types}
\label{fig:tyeff}
\end{figure}
}

\section{Deriving a Type System from the CPS Translation}
\label{sec:type2}

The traditional approach to designing a type system for control operators
is to analyze their CPS translation~\cite{danvy-context}, which exposes
the typing constraints of each syntactic construct.
In the case of \shiftztt and dollar, the CPS translation gives rise to a
stack-like representation of effects, reflecting the ability of \shiftztt
to capture metacontexts~\cite{materzok-subtyping,materzok-hierarchy}.

Unlike control operators, the type system of effect handlers has been
designed independently of the CPS translation.
Indeed, the first type system of effect handlers~\cite{bauer-effsys} was
proposed before the first CPS translation~\cite{hillerstrom-cps}.
This brings up a question: What would we obtain if we derive the type system
of effect handlers from their CPS translation?

In this section, we derive the type system of effect handlers following
the recipe of Danvy and Filinski~\cite{danvy-context}, which consists of
two steps:

\begin{enumerate}
\item Annotate the CPS image of each construct in the most general way
\item Encode the identified typing constraints into the typing rules
\end{enumerate}

\noindent We show that, when guided by the CPS translation, we obtain a
type system with explicit answer types and a restricted form of
answer-type modification~\cite{danvy-context}.
In what follows, we review the existing type systems of \shiftztt/dollar
and effect handlers (Sections~\ref{sec:type2:shiftz} and
\ref{sec:type2:handler}), design a new type system of effect handlers by
applying the CPS approach (Section~\ref{sec:type2:cps}), and prove the
soundness of the type system (Section~\ref{sec:type2:sound}).


\subsection{Type System of Shift0 and Dollar}
\label{sec:type2:shiftz}

In Figure~\ref{fig:lambdasztype}, we present the type system of \lambdasz.
The type system is adopted from Hillerstr\"om et al~\cite{hillerstrom-cps}
(for $\lambda$-terms) and Forster et al.~\cite{forster-jfp} (for \shiftztt
and dollar).
There are two classes of types: value types ($A, B$) and computation types
($C, D$).
The latter take the form $\cty{A}{\epsilon}$, which reads: the computation
returns a value of type $A$ while possibly performing an effect represented
by $\epsilon$\footnote{In the original type system of Forster et
al.~\cite{forster-jfp}, effects are attached to the typing judgment
instead of being part of a computation type.}.
An effect is a list of \textit{answer types}, representing what kind of
context is needed to evaluate a computation\footnote{The effect
representation does not allow \textit{answer-type modification}.
A more general (and involved) type system supporting answer-type
modification is given by Materzok and Biernacki~\cite{materzok-hierarchy}.}.
The list is extended by \rulename{Shift0} (which requires a specific kind
of context) and shrunk by \rulename{Dollar} (which supplies the required
context).

\LambdaszType


\subsection{Type System of Effect Handlers}
\label{sec:type2:handler}

In Figure~\ref{fig:lambdahtype}, we present the type system of \lambdah.
The type system is adopted from Hillerstr\"om et al.~\cite{hillerstrom-cps}.
We again have value and computation types.
An effect is either the pure effect $\pureeff$ or an operation type
$\oparrowty{A}{B}$; the latter tells us what kind of handler is needed to
evaluate a computation.
Operation types are introduced by \rulename{Do} (which requires a specific
kind of operation clause) and eliminated by \rulename{Handle} (which supplies
the required operation clause).
The typing rules for $\lambda$-terms stay the same as those of \lambdasz,
because they do not modify effects.

\LambdahType


\subsection{Applying the CPS Approach}
\label{sec:type2:cps}

The type system in Figure~\ref{fig:lambdahtype} is defined on the 
direct-style expressions in \lambdah.
We now design a new type system based on the CPS translation.
As we saw in Section~\ref{sec:cps}, a CPS-translated \lambdah computation
is a function that receives a return continuation and an effect continuation.
Among the two arguments, a return continuation takes in a value and an 
effect continuation, whereas an effect continuation takes in an operation 
argument and a resumption (a continuation whose handler is already given).
Hence, if we annotate the CPS image of a computation in the most general
way, we obtain something like this:

\im{
\abstwo{k^{\parrowtytwo{A}
                       {(\parrowtytwo{A_1}{(\parrowty{B_1}{C_1})}{D_1})}
                       {E_1}}}
       {h^{\parrowtytwo{A_2}{(\parrowty{B_2}{C_2})}{D_2}}}
       {e^{E_2}}
}

\noindent These annotations are however too general.
The semantics of deep handlers naturally gives rise to the following
invariants.

\begin{enumerate}
\item The types $\parrowtytwo{A_1}{(\parrowty{B_1}{C_1})}{D_1}$ and
  $\parrowtytwo{A_2}{(\parrowty{B_2}{C_2})}{D_2}$ of the two effect 
  continuations are the same.
  This is because a computation and its continuation are evaluated under
  the same handler (when handlers are deep).

\item The return type $E_1$ of the return continuation and the return type
  $C_2$ of the resumption are the same.
  This is because a resumption is constructed using a return continuation.

\item The type $E_2$ of the eventual answer must be either $C_2$ or $D_2$.
  This is because the answer is produced by the return or the operation 
  clause of the surrounding handler.
\end{enumerate}

\noindent These invariants lead to the following refined annotations.

\im{
\abstwo{k^{\parrowtytwo{A}
                       {(\parrowtytwo{\Apr}{(\parrowty{\Bpr}{C})}{D})}
                       {C}}}
       {h^{\parrowtytwo{\Apr}{(\parrowty{\Bpr}{C})}{D}}}
       {e^{E}}
\qquad \text{where } E = C \text{ or } D
}

\noindent The annotations have six different types.
Among them, $\Apr$, $\Bpr$, $C$, $D$, and $E$ specify the kind of the
handler required by the computation, as well as the kind of answer to be  
produced by the handler.
By including these types in non-empty effects, we arrive at the following
effect representation.

\begin{align*}
\epsilon &::= \pureeff \mid \atmeff{A}{B}{C}{D}{E}
\qquad \text{where } E = C \text{ or } D
\end{align*}

\noindent Here, $\atmeff{A}{B}{C}{D}{E}$ carries the following information.

\begin{itemize}
\item The computation may perform an operation of type $\oparrowty{A}{B}$.
\item When evaluated under a handler whose return clause has type $C$ and
  whose operation clause has type $D$, the computation returns an answer
  of type $E$ (which must be either $C$ or $D$).
\end{itemize}

Using this representation of effects, we design the typing rules for
\lambdah computations and values.
We do this by annotating the CPS image of individual syntactic constructs
and converting annotated expressions into typing derivations.

\paragraph{Return Expressions}

We begin by annotating the CPS image of a \returntt expression.
To make the calculation of types easier to follow, we explicitly write the
effect continuation $h$.

\im{
\abstwo{k^{\parrowtytwo{A}
                       {(\parrowtytwo{\Apr}{(\parrowty{\Bpr}{C})}{D})}
                       {C}}}
       {h^{\parrowtytwo{\Apr}{(\parrowty{\Bpr}{C})}{D}}}
    {(\apptwo{k}{\cpsv{V}^A}{h})^C}
}

\noindent The trivial use of the continuation forces the type of the eventual
answer to be $C$ (corresponding to invariant 3 mentioned above).
From this annotated expression, we obtain the following typing rule.

\im{
\typing{\judgev{\Gamma}{V}{A}}
       {\judgeatm{\Gamma}{\ret{V}}{A}{\Apr}{\Bpr}{C}{D}{C}}
       {Return}
}

\paragraph{Application}

We next annotate the CPS image of an application.

\im{
(\app{\cps{V}^{\parrowty{A}{C}}}{\cps{W}^A})^C
}

\noindent The annotations simply tell us that the effect of an application
comes from the body of the function.
This corresponds to the following typing rule.

\im{
\typing{\judgev{\Gamma}{V}{\carrowty{A}{C}} \quad
        \judgev{\Gamma}{W}{A}}
       {\judgec{\Gamma}{\app{V}{W}}{C}}
       {App}
}

\paragraph{Let Expressions}

Having analyzed application, we consider the \lettt expression.

\bigskip

$\lambda k^{\parrowtytwo{B}
           {(\parrowtytwo{\Apr}{(\parrowty{\Bpr}{C})}{D})}
           {C}}.$ \\
\vspace{1mm} \qquad
$\cps{M}^{\parrowtytwo{(\parrowtytwo{A}{(\parrowtytwo{\Apr}{(\parrowty{\Bpr}{C})}{D})}{C})}
                      {(\parrowtytwo{\Apr}{(\parrowty{\Bpr}{C})}{D})}
                      {E}}$ \\
\vspace{1mm} \qquad
$(\abs{x^A}{\app{\cps{N}^{\parrowtytwo{(\parrowtytwo{B}{(\parrowtytwo{\Apr}{(\parrowty{\Bpr}{C})}{D})}{C})}
                                      {(\parrowtytwo{\Apr}{(\parrowty{\Bpr}{C})}{D})}
                                      {C}}}
                {k}})$

\bigskip

\noindent The sequencing of $M$ and $N$ forces the eventual answer type of
$N$ to be $C$.
The application of $\cps{M}$ then forces the eventual answer type of the 
entire construct to be $E$.
The corresponding typing rule is as follows.

\im{
\typing{\judgeatm{\Gamma}{M}{A}{\Apr}{\Bpr}{C}{D}{E} \quad
        \judgeatm{\extenv{\Gamma}{x}{A}}{N}{B}{\Apr}{\Bpr}{C}{D}{C}}
       {\judgeatm{\Gamma}{\letin{x}{M}{N}}{B}{\Apr}{\Bpr}{C}{D}{E}}
       {Let}
}

\paragraph{Operations}

We now move on to the case of an operation call.

\im{
\abstwo{k^{\parrowtytwo{\Bpr}
                       {(\parrowtytwo{\Apr}{(\parrowty{\Bpr}{C})}{D})}
                       {C}}}
       {h^{\parrowtytwo{\Apr}{(\parrowty{\Bpr}{C})}{D}}}
       {(\apptwo{h}{\cpsv{V}}{(\abs{x^A}{\apptwo{k}{x}{h}})})^D}
}

\noindent The application of $h$ forces the eventual answer type to be
$D$ (invariant 3).
The application $\apptwo{k}{x}{h}$ restricts the annotations in two ways:
(i) the handler type of the whole expression and that of $k$ must coincide
(invariant 1); and (2) the return type of $k$ and that of the resumption
must coincide (invariant 2).
Below is the derived typing rule.

\im{
\typing{\judgev{\Gamma}{V}{\Apr}}
       {\judgeatm{\Gamma}{\doop{V}}{\Bpr}{\Apr}{\Bpr}{C}{D}{D}}
       {Do}
}

\paragraph{Handlers}

Lastly, we consider the case of a handler.

\bigskip

$\cps{M}^{\parrowtytwo{(\parrowtytwo{A}{(\parrowtytwo{\Apr}{(\parrowty{\Bpr}{C})}{D})}{C})}
                      {(\parrowtytwo{\Apr}{(\parrowty{\Bpr}{C})}{D})}
                      {E}}$ \\
\vspace{1mm} \hspace*{3mm} $(\abstwo{x^A}{h^{\parrowtytwo{\Apr}{(\parrowty{\Bpr}{C})}{D}}}{\cps{e_r}^{C}})$ \\
\vspace{1mm} \hspace*{3mm} $(\abstwo{x^{\Apr}}{k^{\parrowty{\Bpr}{C}}}{\cps{e_h}^{D}}))^{E}$

\bigskip

\noindent The construction requires the effect of the handled expression $M$
and the type of the two clauses of the handler to be consistent.
Thus we obtain the following typing rule.

\im{
\typing{\begin{trgather}
        \judgeatm{\Gamma}{M}{A}{\Apr}{\Bpr}{C}{D}{E} \\
        \judgec{\extenv{\Gamma}{x}{A}}{M_r}{C} \\
        \judgec{\extenvtwo{\Gamma}{x}{\Apr}{k}{\carrowty{\Bpr}{C}}}
               {M_h}{D}
        \end{trgather}}
       {\judgec{\Gamma}{\handle{M}{x}{M_r}{x}{k}{M_h}}{E}}
       {Handle}
}

\paragraph{Values}

Let us complete the development by giving the typing rules for variables
and abstractions.
These are derived straightforwardly from the CPS translation on values.
The only thing that is non-standard is the well-formedness condition
$\judgety{A}$ in the \rulename{Var} rule.
Here, well-formedness means every function type that appears in a type
takes the form
$\carrowty{A}{\catmty{B}{\Apr}{\Bpr}{C}{D}{C}}$ or
$\carrowty{A}{\catmty{B}{\Apr}{\Bpr}{C}{D}{D}}$.

\im{
\typing{x : A \in \Gamma \quad \judgety{A}}
       {\judgev{\Gamma}{x}{A}}
       {Var}
\qquad
\typing{\judgec{\extenv{\Gamma}{x}{A}}{M}{C}}
       {\judgev{\Gamma}{\abs{x}{M}}{\carrowty{A}{C}}}
       {Abs}
}

Through this exercise, we obtained a type system that is different from
the one presented in Figure~\ref{fig:lambdahtype}.
First, the type system explicitly keeps track of answer types.
Second, the type system allows a form of answer-type
modification~\cite{danvy-context,asai-printf,kobori-atm}.
Answer-type modification is the ability of an effectful computation to
change the return type of its delimited context.
In our setting, this ability is gained by allowing the return and operation
clauses of a handler to have different types.

The answer-type modification supported in this type system is however very
limited.
Specifically, it does not allow answer-type modification in a captured
continuation.
In fact, we have already seen where this restriction comes from: the \lettt
expression.
Recall that, in the typing rule \rulename{Let}, the body $N$ of \lettt
(which corresponds to the continuation of $M$) has a type of the form
$\cty{B}{\atmeff{\Apr}{\Bpr}{C}{D}{C}}$.
This means $N$ must be a pure computation (which is eventually handled by
the return clause of type $C$) or it must be handled by a handler whose
return and operation clauses have the same type (i.e., $C = D$).


\subsection{Soundness}
\label{sec:type2:sound}

The type system we derived from the CPS translation is sound.
We state this property as the following theorem.

\begin{theorem}[Soundness of Type System]
\label{thm:soundtype}
If $\judgec{\Gamma}{M}{\cty{A}{\pureeff}}$, then $M \reduce \ret{V}$,
and $\judgev{\Gamma}{V}{A}$.
\end{theorem}

\begin{proof}
The statement is witnessed by a CPS interpreter\footnote{The interpreter is
available at \url{https://github.com/YouyouCong/tfp22}.}
written in the Agda programming language~\cite{norell-phd}.
The interpreter takes in a well-typed \lambdah computation and returns
a fully-evaluated Agda value.
The signature of the interpreter corresponds exactly to the statement of
soundness, and the well-typedness of the interpreter implies the validity
of the statement.
\end{proof}
