\section{Related Work}
\label{sec:related}

\paragraph{Variations of Effect Handler Calculi}

There is a discussion of the relationship between effect operations with and
without a label, which is similar to our discussion of handlers with and
without the return clause.
The handling of a labeled operation may involve skipping of handlers that
do not take care of the operation in question.
To simulate labeled operations in a calculus with unlabeled operations,
Biernacki et al.~\cite{biernacki-care} introduce an operator $[\_]$
called ``lift'', which allows an operation to skip the innermost handler
and be handled by an outer handler.
This enables programming with multiple effects without using labels,
although it requires the programmer to know what handlers are surrounding
each operation in what order.


\paragraph{CPS Translations of Effect Handlers}

The CPS translations of effect handlers have been developed as a technique
for compiling effect handlers into common runtime platforms.
For this reason, existing CPS translations are either directed by the
specific typing discipline of the source
language~\cite{leijen-cps,brachthauser-mass} or tuned for the particular
implementation strategy of the compiler~\cite{hillerstrom-cps,hillerstrom-jfp}.
We derived a CPS translation of effect handlers from the definitional CPS
translation of \shiftztt and dollar.
As a result, we obtained a translation that is identical to
Hillerstr\"om's~\cite{hillerstrom-cps} unoptimized translation, which we
regard as the definitional translation of general (deep) effect handlers.


\paragraph{Type Systems of Effect Handlers}

There are different flavors of type systems for effect handlers.
In the most traditional type systems~\cite{bauer-effsys,kammar-handler},
effects are represented as a set of operations, similar to the effect
systems for side-effect analysis~\cite{nielson-effsys}.
In some research languages such as Koka~\cite{leijen-koka},
Frank~\cite{lindley-frank}, and Links~\cite{hillerstrom-links}, effects are
treated as row types, which were originally developed for type inference
with records~\cite{remy-record}.
More recently, several languages~\cite{zhang-tunnel,brachthauser-lightweight}
adopt the notion of capabilities from object-oriented programming as an
approach to safe handling of effects.
Unlike the type system we developed in Section~\ref{sec:type2}, these type
systems are all defined independent of the CPS translation, and they do not
explicitly carry answer types.


\paragraph{Effect Handlers and Control Operators}

The relationship between effect handlers and control operators has been 
studied from several different perspectives.
Forster et al.~\cite{forster-jfp} and Pir\'og et al.~\cite{pirog-typedeq} 
establish macro translations to compare the expressiveness of the two 
facilities.
Kiselyov and Sivaramakrishnan~\cite{kiselyov-eff} define a similar 
translation to embed the Eff language into OCaml.
We exploit the results of these studies to enhance our understanding of 
effect handlers.
