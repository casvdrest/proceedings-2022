\newcommand{\Lambdasz}{
\begin{figure}[t]

\lefttext{Syntax}
\begin{align*}
V, W &::= x \mid \abs{x}{M} && \text{Values} \\
M, N &::= \ret{V} \mid \app{V}{V} \mid \letin{x}{M}{M} \mid
       \shiftz{k}{M} \mid \dollarret{M}{x}{M} && \text{Computations}
\end{align*}

\medskip

\lefttext{Evaluation Contexts}
\begin{align*}
E &::= \hole \mid \letin{x}{E}{M} \mid \dollarret{E}{x}{M} 
  && \text{General Contexts} \\
F &::= \hole \mid \letin{x}{F}{M} 
  && \text{Pure Contexts}
\end{align*}

\medskip

\lefttext{Reduction}
\begin{align*}
E[\app{(\abs{x}{M})}{V}]
  &\reduce E[\subst{M}{x}{V}]
  && \rulename{$\beta_v$} \\
  E[\letin{x}{\ret{V}}{M}]
  &\reduce E[\subst{M}{x}{V}]
  && \rulename{$\zeta_v$} \\
  E[\dollarret{\ret{V}}{x}{M}]
  &\reduce E[\subst{M}{x}{V}]
  && \rulename{$\beta_{\$}$} \\
  E[\dollarret{F[\shiftz{k}{M}]}{x}{N}]
  &\reduce E[\subst{M}{k}{\abs{y}{\dollarret{F[\ret{y}]}{x}{N}}}]
  && \rulename{$\beta_{\shiftzsym}$}
\end{align*}

\caption{Syntax and Reduction Rules of \lambdasz}
\label{fig:lambdasz}
\end{figure}
}

\section{\lambdasz: A Calculus of Shift0 and Dollar}
\label{sec:lambdasz}

\setcounter{footnote}{0}

As a calculus of control operators, we consider a minor variation of
Forster et al.'s calculus of \shiftztt and dollar~\cite{forster-jfp},
which we call \lambdasz.
In Figure~\ref{fig:lambdasz}, we present the syntax and reduction rules
of \lambdasz.
The calculus differs from that of Forster et al. in that it is formalized
as a fine-grain call-by-value calculus~\cite{levy-finegrain} instead of
call-by-push-value~\cite{levy-cbpv}.
This means (i) functions are classified as values; and (ii) computations
must be explicitly sequenced using the \lettt expression.
The fine-grain syntax simplifies the CPS translation and type system
developed in later sections.

Among the control constructs, $\shiftz{k}{M}$ (pronounced ``shift'')
captures a continuation surrounding itself.
The other construct $\dollarret{M}{x}{N}$ (pronounced ``dollar'') computes
the main computation $M$ in a delimited context that ends with the
continuation $N$\footnote{The dollar operator was originally proposed by 
Kiselyov and Shan~\cite{kiselyov-substructural}.
In their formalization, the construct takes the form $\dollar{N}{M}$,
where $M$ is the main computation and $N$ is an arbitrary expression
representing an ending continuation.
We use the bracket notation of Forster et al., restricting $N$ to be an 
abstraction $\abs{x}{N}$.
This makes it easier to compare dollar with effect handlers.}.

There are two reduction rules for the control constructs.
If the main computation of dollar evaluates to a value $V$, the whole
expression evaluates to the ending continuation $N$ with $V$ bound to $x$
(rule \rulename{$\beta_\$$}).
If the main computation of dollar evaluates to $F[\shiftz{k}{M}]$, where $F$ 
is a pure evaluation context that has no dollar surrounding a hole, the whole 
expression evaluates to $M$ with $k$ being the captured continuation
$\abs{y}{\dollarret{F[\ret{y}]}{x}{N}}$
(rule \rulename{$\beta_{\shiftzsym}$}).
Notice that the continuation includes the dollar construct that was
originally surrounding the \shiftztt operator.
This design is shared with the \shifttt operator of Danvy and
Filinski~\cite{danvy-abstracting}.
Notice next that the body of \shiftztt is evaluated without being surrounded
by the original dollar.
This differentiates \shiftztt from \shifttt, and allows \shiftztt to capture
a \textit{meta-context}\footnote{To see what a meta-context is, consider the 
following program:

\vspace{-2mm}

\im{\dollarret{2 * 
      \dollarret{1 + 
        \shiftz{k_1}{\shiftz{k_2}{\app{k_2}{3}}}}
      {x}{\ret{x}}}
    {x}{\ret{x}}
    \reduces 6}

\noindent The first \shiftztt operator captures the delimited context 
$\dollarret{1 + \hole}{x}{\ret{x}}$, whereas the second one captures the 
meta-context $\dollarret{2 * \hole}{x}{\ret{x}}$ surrounding the innermost
dollar operator.}, i.e., a context that resides outside of the lexically 
closest dollar.

\Lambdasz
