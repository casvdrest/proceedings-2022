\startcomponent equivalence
\product thesis
\environment thesislayout


\section[sec:contextequiv]{Contextual equivalence}

Our goal is to examine when two \TOPHAT-programs are semantically equivalent.
However, let us first consider what it means in general for two programs $e_1$ and $e_2$ to be semantically equivalent,
denoted by $e_1 \exprequiv e_2$.
According to Sewell \cite[Sewell2009], a good definition of semantic equivalence should satisfy the following properties:

\startitemize[n]
  \item
    Programs that result in values that are observably different should not be equivalent.
  \item
    Programs that terminate should not be equivalent to programs that do not terminate.
  \item
    The relation $\exprequiv$ should be an \emph{equivalence relation}:
    it is reflexive, symmetric and transitive.
  \item
    The relation $\exprequiv$ should be a \emph{congruence}.
    That is if $e_1 \exprequiv e_2$, then for all program contexts $E[\cdot]$,
    we like to have $E[e_1] \exprequiv E[e_2]$.
    Here we fill the hole $\cdot$ with any well typed expression,
  \item
    The relation $\exprequiv$ should contain as many programs as possible subject to the above properties.
\stopitemize

It should be obvious that the first three properties are essential.
Property 4 about congruence states that if two programs $e_1$ and $e_2$ are semantically equivalent,
then we should be able to use $e_1$ and $e_2$ interchangeably within any program without changing its meaning.
Finally, Property 5 ensures that $\exprequiv$ is not just the empty relation.
We will keep these properties in mind when giving our definitions of semantic equivalence.
Throughout this text,
we use the symbol $\exprequiv$ for the semantic equivalence of two expressions in the host language,
and the symbol $\taskequiv$ for the semantic equivalence of two tasks.

% In the following two sections, \see[sec:exprequiv] and \see[sec:taskequiv],
% we will look at when two expressions in the host-language and two tasks are semantically equivalent.
Next, in \see[sec:exprequiv], we give a definition for the semantic equivalence of two expressions in our host language,
the \STLC.
Then, in \see[sec:conditions], we introduce five conditions tasks can be in.
Using these conditions, we define the semantic equivalence of two tasks in \see[sec:taskequiv].
% The next section, Section \see[sec:expreq], will define when we consider two (non-task) expressions equivalent.
% The following two sections, Section \see[sec:expreq] and Section \see[sec:taskeq], will look at when we consider two expressions, and two tasks to be semantically equivalent.
Finally, \see[sec:laws] presents a set of laws that we believe hold true for \TOPHAT-programs with our definition.


\stopcomponent