\startcomponent recap
\product thesis
\environment thesislayout


\doifthesiselse
  {\section{Recap}}
  {\section[sec:conclusions]{Conclusions}}

In this \doifthesiselse{chapter}{paper}, we took \TOPHAT\ as a foundation to reasoning about task-oriented programs.
We gave a definition for the semantic equivalence of two \TOPHAT-programs.
We split this definition into two classes: expression equivalence, and task equivalence.
For task equivalence, we showed that a task can be in either one of five conditions after fixation,
and for every two tasks in the same condition, we defined what it means for them to be semantically equivalent.
We also noted that for task conditions that still accept user input,
it is important to take the interactive setting of \TOPHAT\ into account,
and compare how both tasks react to user input.
We presented a set of transformation laws for \TOPHAT-programs.
Especially, we showed that the $\Task$ type constructor in \TOPHAT\ is not a monad but it is a functor.
Using these laws, developers and compilers can do safe transformations on task-oriented programs,
while being sure the semantic meaning of the program stays the same.


\stopcomponent
