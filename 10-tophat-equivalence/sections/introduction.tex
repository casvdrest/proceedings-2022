\section[sec:introduction]{Introduction}

Task-oriented programming (\TOP) is a relatively new programming paradigm designed for developing distributed interactive multi-user workflow systems.
In this programming paradigm, the concept of a \emph{task} plays a central role.
A task is a unit of work assigned to some user, and consists of two parts: a description of the work that should be done, and a typed interface that defines the type of the task value that it returns.
Tasks are described in an abstract, declarative manner, and from this abstract description, a \TOP\ engine automatically generates a \GUI.
It also takes care of the client-server communication and user authentication that is needed for users to work together on tasks.
\TOP\ allows programmers to define workflows which describe what tasks should be executed by its users, without having to worry about how this is achieved.


% \paragraph{Collaboration in TOP}

User collaboration is a central concept in \TOP.
The different ways in which users can collaborate are captured by \emph{task combinators}.
By using these combinators, \TOP-programmers can construct larger tasks from smaller ones in several ways.
There is \emph{sequential composition}, which allows tasks to be executed one after the other.
And there is \emph{parallel composition}, which allows tasks to be executed in parallel at the same time.
For parallel composition, it is possible to either combine the results, or to conditionally continue with either one of two tasks.

In order to collaborate, users also need to be able to communicate with each other, and with the system.
Using task combinators, it is possible to pass along \emph{data} from one task onto the next.
For communication with the outside world, there are \emph{editors}.
They provide interaction with the environment via \emph{input} events.
Editors are typed containers which remember the last value that has been sent to them, and users can communicate with the system through these editors.
Furthermore, editors allow users to view and edit \emph{shared data}, which are mutable references whose changes are immediately visible to all other tasks watching them.

Another important element of \TOP\ is that task are \emph{typed}.
This is important to determine the type of the values that are communicated to other tasks and to the environment.
Not all tasks have a value, and a task does not just produce a value when it is completed.
Instead, a task's value is continually updated while the work takes place, and can be observed at any point during execution.
Moreover, it may be possible that a task is never completed, and that its value never reaches a stable condition.
A task's value reflects a task's current progress.
They can be inspected by other tasks to base decisions on, which in turn can impact the things users can see or do.

Finally, the \TOP\ language is \emph{modular}:
tasks are composed of smaller tasks, and can be arguments to or results of functions.
This allows programmers to re-use tasks, and to model their own collaboration patterns using higher-order tasks \cite[conf/ppdp/PlasmeijerLMAK12].


\subsubject{Related work}
% \paragraph{TOP engines}

\TOP\ describes in an abstract way \emph{what} work should be done by the system and its users.
It does not describe \emph{how} this should be done,
this question should be answered by a \emph{\TOP\ engine}.
The \ITASKS\ engine \cite[conf/cefp/AchtenKP13] is an implementation of \TOP, written in the pure and lazy functional programming language \CLEAN\ \cite[plasmeijer2002clean].
It is implemented as a shallowly embedded domain-specific language, which means that it inherits features from its host language \CLEAN.
Amongst these features is a strong typing system, and because \CLEAN\ is a functional language, it allows task combinators to be expressed as functions.
From the high level description of the tasks, \ITASKS\ generates a web application that is able to execute the described tasks.
It takes care of generating a \GUI, and of coordinating the tasks in a distributed manner by using a client-server architecture.
The server side of an \ITASKS\ application runs a web service to which users on a wide range of different devices can connect, and the client side realises the front-end components.
This way, programmers of iTasks-programs need not be concerned by lower-level implementation details.
\ITASKS\ has shown itself effective in the past for the implementation of interactive, distributed, workflow applications \cite[conf/iscram/LijnseJP12,conf/sfp/StutterheimAP17].


% \paragraph{TOP formalisations}

Because \ITASKS\ has been designed for developing real-world applications, reasoning formally about \ITASKS\ programs is hard \cite[conf/ifl/KoopmanPA08].
Steenvoorden et.al \cite[conf/ppdp/SteenvoordenNK19] introduces \TOPHAT, a formal language plus operational semantics for reasoning about task-oriented programs.
% A follow up paper, \emph{TopHat Next: even more stylish task-oriented programming}, is currently being written \cite[TopHatNext].
% This thesis uses these two papers as starting point.
The semantics of \TOPHAT\ have been machine verified\footnote{\url{https://github.com/timjs/tophat-proofs}}
using the dependently typed programming language \IDRIS\ \cite[journals/jfp/Brady13].
These formal semantics have been used to prove user defined properties of tasks using symbolic execution \cite[conf/ifl/NausSK19].
An example of this are the circumstances when a a tax subsidy request for solar panels should be granted.
Symbolic execution has also been applied to generate next step hints for end users \cite[conf/sfp/NausS20].
For example, does the given data during a subsidy request indeed lead to subsidy greater then zero?


\subsubject{Motivation}

% However,
When two tasks are equivalent is a long standing open research question.
The formal semantics of \TOPHAT\ are a good starting point to investigate this question.
\TOPHAT\ is a domain specific language embedded in the \STLC.
The semantic equivalence in context of the (simply typed) $\lambda$-calculus has been extensively studied \cite[conf/ac/Pitts00,Sewell2009].
As \TOPHAT\ is build on top of the \STLC,
we can relay on this work for equivalence in the host language.
However, this knowledge is not immediately transferable to the world of tasks.
There, we need to take the semantics of task constructs into account.

\startexample[exm:task-absolute-value]{Task equivalence}
  The next two tasks both ask end users to enter an integer (using $\Enter$),
  and thereafter (using $\Step$) show the absolute value of the given integer (using $\View$):
  \startformulas[align]
    \NC t_1 \NC:=\NC \Enter\Int \Step \lambda x:\Int.\ \Ite{x < 0}{\View (-x)}{\View x}    \NR
    \NC t_2 \NC:=\NC \Enter\Int \Step \lambda y:\Int.\ \Ite{y \geq 0}{\View y}{\View (-y)} \NR
  \stopformulas
  It should be clear that these two tasks are equivalent.
\stopexample

% Our research question is the following:

% \startquotation
%   When are two programs in \TOPHAT\ semantically equivalent?
% \stopquotation

If we can define such a notion of semantic equivalence for task-oriented programs, what interesting properties can we prove (or disprove)?
If we know that one task-oriented program is semantically equivalent to another, then we know that we can substitute one for the other, without changing the meaning of the program,
which in turn could be useful for doing compiler optimizations.
An example of such optimisations is whether the monad laws hold for the step combinator ($\Step$) used in above example.
Showing that certain equalities hold for task-oriented programs could also prove useful for the \ITASKS\ system in the future.


\subsubject{Structure}

The remainder of this paper is structured as follows.
\see[sec:tophat] will explain the components of the \TOPHAT\ language,
together with some examples.
We will follow the updated semantics presented in \cite[Steenvoorden22].
%
% In the following two sections, \see[sec:exprequiv] and \see[sec:taskequiv],
% we will look at when two expressions in the host-language and two tasks are semantically equivalent.
In the following section, \see[sec:contextequiv], we give a short overview of what contextual equivalence means.
Then, in \see[sec:exprequiv], we give a definition for the semantic equivalence of two expressions in our host language,
and in \see[sec:conditions] we show that a task can be in either one of five conditions.
For every two tasks in the same condition, we define what it means for them to be semantically equivalent in \see[sec:taskequiv].
Additionally, \see[sec:laws] presents a set of laws that we believe hold true or false for \TOPHAT-programs according to our definition.
Finally, \see[sec:conclusions] will conclude this paper.
