% Context
Task-oriented programming (\TOP) is a new programming paradigm for specifying multi-user workflows.
To reason formally about \TOP\ programs, a formal language called \TOPHAT\ has been designed, together with its operational semantics.
% Inquiry
For proving properties about task-oriented programs, it is desirable to know when two \TOPHAT-programs are semantically equivalent.
This paper aims to answer this question.
% Approach
We show that a task can be in either one of five conditions,
and for every two tasks in the same condition, we define what it means for them to be semantically equivalent.
% Knowledge
Using this definition, we study a number of transformation laws for \TOPHAT-programs,
which can be used by developers and compilers to optimise \TOP-programs.
% Importance
We show that the $\Task$ operation on types in \TOPHAT\ is a functor but cannot be a monad.
% Grounding
We support our findings with proofs formalised in the dependently typed programming language \IDRIS.

% Context: What is the broad context of the work? What is the importance of the general research area?
% Inquiry: What problem or question does the paper address? How has this problem or question been addressed by others (if at all)?
% Approach: What was done that unveiled new knowledge?
% Knowledge: What new facts were uncovered? If the research was not results oriented, what new capabilities are enabled by the work?
% Grounding: What argument, feasibility proof, artefacts, or results and evaluation support this work?
% Importance: Why does this work matter?
