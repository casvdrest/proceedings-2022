\startcomponent laws
\product thesis
\environment thesislayout


\section[sec:laws]{Laws on tasks}

\definetabulate[laws][|rM|cM|lM|r|]

Now that we have a definition of semantic equivalence for \TOPHAT-programs, we can look at some interesting properties.
This section gives some transformation properties that hold (or not) for task equivalence.

Whenever we write a task equivalence, we implicitly universally quantify over tasks $t$.
So $t_1 \taskequiv t_2$, where tasks $t_1$ and $t_2$ contain task $t$, should be read as $\forall t:\Task\;\tau.\ t_1 \taskequiv t_2$.
Moreover, whenever we write a task inequivalence, then we mean that there exists at least one task $t$ for which the equivalence does not hold.
So $t_1 \ntaskequiv t_2$, where tasks $t_1$ and $t_2$ contain task $t$, should be read as $\neg\forall t:\Task\;\tau.\ t_1 \taskequiv t_2$, or $\exists t:\Task\;\tau.\ t_1 \ntaskequiv t_2$.
For some (in)equalities we may use additional tasks, expressions, or functions, in which case these (in)equalities should be interpreted in the same way.


% \subsection{Transform}
%
% A functor in functional programming and category theory is a mapping.
% In the functional programming formulation, it consists of:
% \startitemize[compact]
%   \item a type constructor $f$;
%   \item a map operation that, given a function $a \to b$ gives a function $f\, a \to f\, b$.
% \stopitemize
% In \TOPHAT, we can map a function over a task by using the transform ($\Trans$) combinator.
% %
% An object $f$ is a functor if it satisfies:
% \startitemize[n]
%   \item the \emph{identity law}, which says that if we map the identity function over $f\, a$, then the result is again $f\, a$; and
%   \item the \emph{composition law}, which says that mapping the composition of two functions over $f$ is the same as mapping the two functions over $f$ in a row.
% \stopitemize

In \TOPHAT, we can express functor laws over the type constructor $\Task$ as follows.
\startproposition[prp:transforming-laws]{Laws on transform ($\Trans$)}
  \startformulas[tag]
    \NC \Id \Trans t
    \NC \taskequiv
    \NC t
    \NC \text{identity}
  \TR
    \NC (e_1 \circ e_2) \Trans t
    \NC \taskequiv
    \NC e_1 \Trans\ (e_2 \Trans t)
    \NC \text{composition}
  \NR
  \stopformulas
\stopproposition

\startproofs
  Given any fixated task $t$ and state $\sigma$, we have that
  \startitemize
    \item $\Failing(\Id \Trans t) = \Failing(t)$ by definition (see \doifthesiselse{\see[fig:observation-failing]}{\see[fig:observation-failing-inputs]});
    \item $\Inputs(\Id \Trans t) = \Inputs(t)$ by definition (see \doifthesiselse{\see[fig:observation-input]}{\see[fig:observation-failing-inputs]}); and
    \item $\Value(\Id \Trans t, \sigma) = \Value(t, \sigma)$,
      because if $t$ has no value, then neither does \TS{id <*> t},
      and if $\Value(t,\sigma) = v$ for $v \neq \bot$, then $\Value(\Id \Trans t, \sigma) = \Id\ v \evaluate v$.
  \stopitemize
  A similar argumentation can be made for the composition law.
  Therefore the $\Task$ operator on types is a functor.
\stopproofs


% \subsection{Pairing}
%
% A monoidal functor is a construct in functional programming and category theory lying between a functor and a monad.
% Informally, it is a functor which preserves the monoidal structure of the category.
% In functional programming it consists of
% \startitemize
%   \item a type constructor $f$;
%   \item a functorial unit value $f\, \tuple{}$,
%     which provides a default value whose context wraps nothing of interest;
%   \item a pairing operation that, given two values $f\, a$ and $f\, b$,
%     pairs both functorial values in a new value $f\, \tuple{a,b}$.
% \stopitemize
% On the type constructor $\Task$, these are the lift ($\Lift$) and pair ($\Pair$) combinators respectively.
% An alternative formulation is using a pure and apply function, as in Haskell's applicative type class.

For the \TOPHAT\ pairing construct ($\Pair$),
we can formulate laws which are inspired by the monoidal functor formulation of the laws of Haskell's Applicative type class \cite[journals/jfp/McbrideP08].
% The laws that a monoidal functor must satisfy can be written in \TOPHAT\ as follows:
\startproposition[prp:pairing-laws]{Laws on pair ($\Pair$)}
  \startformulas[tag]
    \NC \Lift \unit \Pair t
    \NC \ntaskequiv
    \NC t
    \NC \text{left identity}
  \TR
    \NC t \Pair \Lift \unit
    \NC \ntaskequiv
    \NC t
    \NC \text{right identity}
  \TR
    \NC \Assoc \Trans (t_1 \Pair\ (t_2 \Pair t_3))
    \NC \taskequiv
    \NC (t_1 \Pair t_2) \Pair t_3
    \NC \text{associativity}
  \TR
    \NC \tuple{e_1\ \_, e_2\ \_} \Trans\ (t_1 \Pair t_2)
    \NC \taskequiv
    \NC (e_1 \Trans t_1) \Pair\ (e_2 \Trans t_2)
    \NC \text{naturality}
  \NR
  \stopformulas
\stopproposition

\startproofs
  If we take $t = \Fail$ for the left and right identity laws,
  then in both cases we have that the right-hand side is a failing task, but the left-hand side is stuck.
  Pairing requires that both sides fail before it fails itself.
  Therefore, both the left and right identity laws do not hold.
  %the $\Task$ constructor on types does not form a monoidal functor.

  However, the other two laws hold using a similar argumentation as in the proof of \see[prp:transforming-laws].
  For associativity, we need the function $\Assoc := \lambda \tuple{a, \tuple{b,c}}. \tuple{\tuple{a,b},c}$ to make sure that both sides have the same type.
  Also, the naturality law states that first pairing two tasks and then mapping two functions is the same as first mapping the functions separately and then pairing them.
\stopproofs


% \subsection{Choosing}

For choosing ($\Choose$) we formulate the following properties,
which are loosely based on the laws of the Alternative type class in Haskell.
\startproposition[prp:choosing-laws]{Laws on choose ($\Choose$)}
  \startformulas[tag]
    \NC t_1 \Choose\ (t_2 \Choose t_3)
    \NC \taskequiv
    \NC (t_1 \Choose t_2)\ \Choose t_3
    \NC \text{associativity}
  \TR
    \NC \Lift e \Choose t
    \NC \taskequiv
    \NC \Lift e
    \NC \text{left catch}
  \TR
    \NC t \Choose \Lift e
    \NC \ntaskequiv
    \NC \Lift e
    \NC \text{right catch}
  \TR
    \NC e \Trans\ (t_1 \Choose t_2)
    \NC \taskequiv
    \NC (e \Trans t_1)\ \Choose\ (e \Trans t_2)
    \NC \text{distributivity}
  \TR
    \NC t_0 \Pair\ (t_1 \Choose t_2)
    \NC \ntaskequiv
    \NC (t_0 \Choose t_1) \Pair\ (t_0 \Choose t_2)
    \NC \text{left pair}
  \TR
    \NC (t_1 \Choose t_2) \Pair t_0
    \NC \ntaskequiv
    \NC (t_1 \Choose t_0) \Pair\ (t_2 \Choose t_0)
    \NC \text{right pair}
  \TR
    \NC t_0 \Step \lambda x. (t_1 \Choose t_2)
    \NC \taskequiv
    \NC (t_0 \Step \lambda x. t_1)\ \Choose\ (t_0 \Step \lambda x. t_2)
    \NC \text{left step}
  \TR
    \NC (t_1 \Choose t_2)\ \Step e
    \NC \taskequiv
    \NC (t_1 \Step e) \Choose\ (t_2 \Step e)
    \NC \text{right step}
  \NR
  \stopformulas
\stopproposition

\startproofs
  Associativity, left catch, and a number of distributivity laws can be proved in a similar way as in \see[prp:transforming-laws].
  Catch only holds for the left side, because the choice combinator is left-biased.
  The first distributivity law shows that mapping a function over choice should be equivalent to mapping it separately over its component subtasks.
  The last two distributivity laws for step ($\Step$) show that stepping to or from a choice is equivalent to choosing between steps.

  Left and right distributivity for pairing ($\Pair$) do not hold.
  Suppose $\Value(t_0,\sigma) = \bot$, $\Value(t_1,\sigma) = v_1$ and $\Value(t_2,\sigma) = v_2$. % for $v_1 \neq \bot$ and $v_2 \neq \bot$.
  Then in both cases, we have that the left-hand side of the inequality has no value, because pairing requires that both sides have a value before it has a value itself.
  However, the right-hand side does have a value in both cases, namely $\tuple{v_1,v_2}$, because the choice combinator normalises $t_0$ away, and we are left with $t_1 \Pair t_2$ in both cases.
\stopproofs


% \subsection{Failing}

When considering the failing task ($\Fail$), we can formulate the following laws:
\startproposition[prp:_]{Laws on fail ($\Fail$)}
  \startformulas[tag]
    \NC \Fail \Choose t
      \NC \taskequiv
      \NC t
      \NC \text{left identity}
      \TR
    \NC t \Choose \Fail
      \NC \taskequiv
      \NC t
      \NC \text{right identity}
      \TR
    \NC e \Trans \Fail
      \NC \taskequiv
      \NC \Fail
      \NC \text{annihilation}
      \TR
    \NC \Fail \Pair t
      \NC \ntaskequiv
      \NC \Fail
      \NC \text{left zero}
      \TR
    \NC t \Pair \Fail
      \NC \ntaskequiv
      \NC \Fail
      \NC \text{right zero}
      \TR
    \NC \Fail \Step e
      \NC \taskequiv
      \NC \Fail
      \NC \text{left annihilation}
      \TR
    \NC t \Step \lambda x. \Fail
      \NC \ntaskequiv
      \NC \Fail
      \NC \text{right annihilation}
      \NR
  \stopformulas
\stopproposition

\startproofs
  The failure task ($\Fail$) acts as the left and right identity for the choice combinator ($\Choose$).
  That means that\~$\Fail$ can be cancelled out from the left- and right-hand side of\~$\Choose$.
  For pairing, left and right zero do not hold, because by definition $\Failing(t_1 \Pair t_2) = \Failing(t_1) \wedge \Failing(t_2)$ (see \doifthesiselse{\see[fig:observation-failing]}{\see[fig:observation-failing-inputs]}).
  That is, pairing requires that both the left-hand and right-hand side fail before their pairing fails.
  Thus if only one side fails, it cannot be equivalent to the failing task\~$\Fail$.

  We defined all failing tasks to be semantically equivalent, thus annihilation for $\Trans$, and left step annihilation for $\Step$ trivially hold.
  Right step annihilation does not however.
  Suppose that $t$ is not a failing task, then we also have that $t \Step \lambda x.\Fail$ is not a failing task, because $\Failing(t \Step \lambda x.\Fail) = \Failing(t)$ by definition (see \doifthesiselse{\see[fig:observation-failing]}{\see[fig:observation-failing-inputs]}).
  Neither can we normalise $t \Step \lambda x.\ \Fail$, because the right-hand side fails \doifthesis{(\seerule{N-StepFail})}.
  So if $t$ does not fail, then neither does $t \Step \lambda x.\Fail$, and we cannot conclude that a non-failing task is semantically equivalent to the failing task $\Fail$.
\stopproofs


% \subsection{Step}
%
% A monad in functional programming consists of:
% \startitemize
%   \item a type constructor $m$;
%   \item a return function that wraps any value $a$ into the monadic value $m\ a$; and
%   \item a bind function that extracts the value $a$ from $m\ a$, and uses it to produce a new monadic value $m\ b$.
% \stopitemize
% Furthermore, for type constructor $m$ to be considered a monad, it must satisfy three laws:
% \startitemize[n]
%   \item return is a \emph{left identity} for bind;
%   \item return is a \emph{right identity} for bind; and
%   \item bind is \emph{associative}.
% \stopitemize

Steps ($\Step$) in \TOPHAT\ have a monadic flavour to them, and we can wonder whether the step combinator is a bind operation.
So, if we consider $\Task$ to be the monadic type constructor, $\Lift$ the return function, and $\Step$ the bind function, then we can express the three monadic laws in \TOPHAT\ as follows.
\startproposition[prp:_]{Laws on step ($\Step$)}
  \startformulas[tag]
    \NC \Lift x \Step g
      \NC \ntaskequiv
      \NC g\ x
      \NC \text{left identity}
      \TR
    \NC t \Step\ (\lambda y. \Lift y)
      \NC \ntaskequiv
      \NC t
      \NC \text{right identity}
      \TR
    \NC (t \Step g) \Step h
      \NC \taskequiv
      \NC t \Step\ (\lambda y. g\ y \Step h)
      \NC \text{associativity}
      \NR
  \stopformulas
\stopproposition

\startproofs
  With our definition of program equivalence, we can show that neither the left nor the right identity laws hold.
  For the left identity, take for example $g := \lambda x.\Fail$, then we have that $g\ x$ is a failing task, because $g\ x = (\lambda x.\Fail)\ x \evaluate \Fail$.
  However, the left-hand side $\Lift x \Step g = \Lift x \Step \lambda x.\Fail$ is a stuck task, because it does not fail, and the step can never be taken.

  For the right identity law we can take for example $t := \Enter^k \Int$.
  If we send the input event $k!42$ to both sides, then the left-hand side normalises to $\Lift 42$, while the right-hand side becomes $\Update^k 42$.
  As we have already seen, these two tasks are not semantically equivalent, because the value of $\Lift 42$ is steady and cannot be changed any more,
  while the value of $\Update^k 42$ is unsteady and can be updated through user input.

  However, the associativity law holds.
  Given any fixated task $t$ and state $\sigma$, for $\Failing$ and $\Inputs$ we have that:
  \startformulas[four]
    \NC \Failing\big((t \Step g) \Step h\big) \NC = \Failing(t \Step g) \NC = \Failing(t) \NC = \Failing\big(t \Step (\lambda y.g\ y \Step h)\big) \TR
    \NC \Inputs\big((t \Step g) \Step h\big)  \NC = \Inputs(t \Step g)  \NC = \Inputs(t)  \NC = \Inputs\big(t \Step (\lambda y.g\ y \Step h)\big)  \NR
  \stopformulas
  \noindentation
  by definition (see \doifthesiselse{\see[fig:observation-failing] and \see[fig:observation-input]}{\see[fig:observation-failing-inputs]}).
  Furthermore, supposing that $\Value(t,\sigma) = v$, % for $v \neq \bot$,
  and $g\ v \evaluate t'$ with $\neg\Failing(t')$, then both sides normalise to $t' \Step h$ in some state $\sigma'$.
  On the other hand, if either $\Value(t,\sigma) = \bot$ or $\Failing(t')$, then the step cannot be taken, and we have that $\Value\big((t \Step g) \Step h\big) = \Value\big(t \Step (\lambda y.g\ y \Step h)\big) = \bot$.
\stopproofs

\stopcomponent