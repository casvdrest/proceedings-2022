\startcomponent tasks
\product thesis
\environment thesislayout


\section[sec:taskequiv]{Task equivalence}

\see[fig:task-states] shows all possible task states and their transitions.
In this diagram, a looping task is a task which never leaves the running state,
that is, which always takes the transition back to the running state, no matter what input is given.
A branching task is a running task for which there exists at least one input sequence which will transition to either stuck, steady, or unsteady.
We will argue that this state diagram is complete.

\doifthesiselse{
\placefigure[][fig:task-states]
  {Possible task conditions and their transitions, where a transition is caused by user interaction ($\interact{\iota}$) for some input $\iota$.}
  {\startpicture[shorten >=1pt,minimum size=3.5pc,node distance={0.66pc and 3pc},>={Triangle},draw=black,thick]
    \node[circle,draw,very thick] (running)                            {\emph{Running}};
    \node[circle,draw,very thick] (failing)   [above left=of running]  {\emph{Failing}};
    \node[circle,draw,very thick] (stuck)     [above right=of running] {\emph{Stuck}};
    \node[circle,draw,very thick] (finished)  [below left=of running]  {\emph{Steady}};
    \node[circle,draw,very thick] (recurring) [below right=of running] {\emph{Unsteady}};
    %
    % \draw[thick,->] (running) -- (failing);
    \draw[->] (running) -- (finished);
    \draw[->] (running) -- (recurring);
    \draw[->] (running) -- (stuck);
    \draw[->] (recurring) edge [out=423,in=387,looseness=6]  (recurring);
    \draw[->] (running) edge [out=108,in=72,looseness=6]  (running);
  \stoppicture}
}{\placefigure[][fig:task-states]
  {Possible task conditions and their transitions, where a transition is caused by user interaction ($\interact{\iota}$) for some input $\iota$.}
  {\startpicture[x=1pc,y=1pc,shorten >=1pt,minimum size=3.5pc,node distance={0.33pc and 3pc},>={Triangle},draw=black,thick]
    \node[circle,draw,very thick] (running)                            {\emph{Running}};
    \node[circle,draw,very thick] (failing)   [above left=of running]  {\emph{Failing}};
    \node[circle,draw,very thick] (stuck)     [above right=of running] {\emph{Stuck}};
    \node[circle,draw,very thick] (finished)  [below left=of running]  {\emph{Steady}};
    \node[circle,draw,very thick] (recurring) [below right=of running] {\emph{Unsteady}};
    %
    % \draw[thick,->] (running) -- (failing);
    \draw[->] (running) -- (finished);
    \draw[->] (running) -- (recurring);
    \draw[->] (running) -- (stuck);
    \draw[->] (recurring) edge [out=423,in=387,looseness=6]  (recurring);
    \draw[->] (running) edge [out=108,in=72,looseness=6]  (running);
  \stoppicture}
}

First, we show that the five conditions discussed so far are mutually exclusive and exhaustive.
\startproposition[prp:conditions-exclusive]{Condition exclusivity}
  For all well typed fixated tasks $t$ and states $\sigma$,
  % that is $\Gamma,\Sigma \infers n : \Task\tau$ and $\Gamma,\Sigma \infers \sigma$,
  % thus $t,\sigma,\delta \fixate n,\sigma'$ for some well typed task $t$,
  we have that $t$ is in one of the five conditions show in \see[fig:task-states].
\stopproposition

\startproof
  By \see[prp:failing-no-interaction], we know that if a task is failing, it has no observable value and no possible inputs.
  This makes \cnd{Failing} the only condition a task can be in when $\Failing(t) = \True$.
  Otherwise, one of the four remaining conditions (\cnd{Steady}, \cnd{Unsteady}, \cnd{Running}, and \cnd{Stuck}) should hold.
  These four conditions are mutual exclusive by definition.
  % When none of the observations check, a task is \cnd{Stuck}.
  %
  \doifthesis{This proof is supported by the covering function for the condition view in \impl{Task.Condition}.}
\stopproof

Now that we know the conditions in \see[fig:task-states] are the only conditions for any fixated task to be in,
we show that the transitions in the state diagram are also complete.

\startproposition[prp:task-states]{Condition completeness}
  The state diagram in \see[fig:task-states] is complete.
  That is, for any two task conditions $C$ and $C'$,
  there is a transition from $C$ to $C'$
  if and only if there exist two tasks $t \in C$ and $t' \in C'$ such that
  $t,\sigma \interact{\iota} t',\sigma'$ for some input $\iota \in \Inputs(t)$ and states $\sigma$ and $\sigma'$.
\stopproposition

\startproof
  For the conditions in the state diagram of \see[fig:task-states] we have:
  \startitemize
    % \item There are no transitions from \emph{Failing},
    %   because failing tasks do not accept any input by \see[def:failing-tasks].
    % \item The same holds for \emph{Stuck}, by \see[def:stuck-tasks].
    % \item By
    \item Running tasks can \emph{branch} to all other conditions,
      except for the failing condition, as stated in \see[prp:running-not-to-failing],
      or \emph{loop} to itself.
    \item Failing tasks stay failing, by \see[prp:failing-stays-failing].
    \item Stuck tasks stay stuck, by \see[cor:stuck-stays-stuck].
    \item Finished tasks stay finished, by \see[prp:finished-stays-finished].
      Moreover, they keep their input space.
      So steady tasks stay steady, because their input space stays empty,
      and unsteady tasks stay unsteady, keeping their non-empty input space.
  \stopitemize
  Hereby, we covered every possible transition in \see[fig:task-states].
\stopproof

In the previous section, we mentioned that with the addition of internal editors ($\Lift$) and read-only editors ($\View,\Watch$) in this \doifthesiselse{thesis}{paper},
it is no longer the case that a task $t$ is failing if and only if it accepts no more user input,
as is shown in Theorem\~6.5 in the original \TOPHAT\ paper \cite[conf/ppdp/SteenvoordenNK19].
Were this theorem still true, we could not have the stuck and steady states,
because they contain tasks that do not fail and do not accept user input.
We will therefore claim that, if we remove $\Lift$, $\View$, and $\Watch$ from the \TOPHAT\ language as presented here,
we no longer have the stuck and steady conditions.

\startcorollary[cor:done-removal]{Read-only tasks}
  If we remove the editors $\Lift$, $\View$, and $\Watch$ from \TOPHAT,
  we are left with only three possible task conditions after fixation: failing, running and finished unsteady.
\stopcorollary

Based on the five task conditions, we give the following definition for the semantic equivalence of two tasks:
\startdefinition[def:task-eq]{Task equivalence}
  Given two fixated tasks $t_1, t_2 : \Task\ \beta$ of basic inner type, and given a state $\sigma$,
  we say that $t_1$ and $t_2$ are semantically equivalent
  if one of the following holds:
  \startitemize[n]
    \item
      both tasks are \emph{failing};
    \item
      both tasks are \emph{finished} with $\Value(t_1,\sigma) = \Value(t_2,\sigma)$, and this value is \emph{steady};
    \item
      both tasks are \emph{finished} with $\Value(t_1,\sigma) = \Value(t_2,\sigma)$, and this value is \emph{unsteady},
      and for all valid input sequences $I$:
      \NL $t_1,\sigma \interact{I} t_1',\sigma' \iff t_2,\sigma \interact{I} t_2',\sigma'$
      \NL with $\Value(t_1',\sigma') = \Value(t_2',\sigma')$;
    \item
      both tasks are \emph{stuck};
    \item
      both tasks are \emph{running}, and for all valid input sequences $I$:
      \NL $t_1,\sigma \interact{I} t_1',\sigma' \iff t_2,\sigma \interact{I} t_2',\sigma'$
      \NL with $\Value(t_1',\sigma') = \Value(t_2',\sigma')$ and $\Inputs(t_1') = \Inputs(t_2')$.
  \stopitemize
  We write $t_1 \taskequiv t_2$.
\stopdefinition

To determine a task's condition, we use the task observation functions $\Failing$, $\Value$, and $\Inputs$.
For some task conditions, we also need to take the interactive setting of \TOPHAT\ into account.
Namely, if the set of inputs given by $\Inputs$ is not empty,
as is the case for the unsteady and running task states,
we also need to check the task observations after every possible input sequence.

Because we also allow input sequences to be empty, we can generalise the above definition as follows:
\startdefinition[def:task-eq-gen]{Generalised task equivalence}
  Given two fixated tasks $t_1, t_2 : \Task\ \beta$ of basic inner type, and given a state $\sigma$,
  we say that $t_1$ and $t_2$ are semantically equivalent ($t_1 \taskequiv t_2$)
  if for all valid input sequences $I$ we have that
  \NL $t_1,\sigma \interact{I} t_1',\sigma' \iff t_2,\sigma \interact{I} t_2',\sigma'$
  \NL with $\Failing(t_1') = \Failing(t_2')$,  $\Value(t_1',\sigma') = \Value(t_2',\sigma')$, and $\Inputs(t_1') = \Inputs(t_2')$.
\stopdefinition
% \startconjecture
%   \see[def:task-eq2] is equivalent to \see[def:task-eq].
%   That is, they describe the same relation:
%   $$t_1 \taskequiv t_2 \text{ by \see[def:task-eq] } \iff t_1 \taskequiv t_2 \text{ by \see[def:task-eq2]} $$
% \stopconjecture

\stopcomponent